\documentclass[]{article}
\usepackage{lmodern}
\usepackage{amssymb,amsmath}
\usepackage{ifxetex,ifluatex}
\usepackage{fixltx2e} % provides \textsubscript
\ifnum 0\ifxetex 1\fi\ifluatex 1\fi=0 % if pdftex
  \usepackage[T1]{fontenc}
  \usepackage[utf8]{inputenc}
\else % if luatex or xelatex
  \ifxetex
    \usepackage{mathspec}
  \else
    \usepackage{fontspec}
  \fi
  \defaultfontfeatures{Ligatures=TeX,Scale=MatchLowercase}
\fi
% use upquote if available, for straight quotes in verbatim environments
\IfFileExists{upquote.sty}{\usepackage{upquote}}{}
% use microtype if available
\IfFileExists{microtype.sty}{%
\usepackage{microtype}
\UseMicrotypeSet[protrusion]{basicmath} % disable protrusion for tt fonts
}{}
\usepackage[margin=1in]{geometry}
\usepackage{hyperref}
\hypersetup{unicode=true,
            pdftitle={Summarize Peptide Level Measurements},
            pdfauthor={WEW@FGCZ.ETHZ.CH},
            pdfborder={0 0 0},
            breaklinks=true}
\urlstyle{same}  % don't use monospace font for urls
\usepackage{longtable,booktabs}
\usepackage{graphicx,grffile}
\makeatletter
\def\maxwidth{\ifdim\Gin@nat@width>\linewidth\linewidth\else\Gin@nat@width\fi}
\def\maxheight{\ifdim\Gin@nat@height>\textheight\textheight\else\Gin@nat@height\fi}
\makeatother
% Scale images if necessary, so that they will not overflow the page
% margins by default, and it is still possible to overwrite the defaults
% using explicit options in \includegraphics[width, height, ...]{}
\setkeys{Gin}{width=\maxwidth,height=\maxheight,keepaspectratio}
\IfFileExists{parskip.sty}{%
\usepackage{parskip}
}{% else
\setlength{\parindent}{0pt}
\setlength{\parskip}{6pt plus 2pt minus 1pt}
}
\setlength{\emergencystretch}{3em}  % prevent overfull lines
\providecommand{\tightlist}{%
  \setlength{\itemsep}{0pt}\setlength{\parskip}{0pt}}
\setcounter{secnumdepth}{0}
% Redefines (sub)paragraphs to behave more like sections
\ifx\paragraph\undefined\else
\let\oldparagraph\paragraph
\renewcommand{\paragraph}[1]{\oldparagraph{#1}\mbox{}}
\fi
\ifx\subparagraph\undefined\else
\let\oldsubparagraph\subparagraph
\renewcommand{\subparagraph}[1]{\oldsubparagraph{#1}\mbox{}}
\fi

%%% Use protect on footnotes to avoid problems with footnotes in titles
\let\rmarkdownfootnote\footnote%
\def\footnote{\protect\rmarkdownfootnote}

%%% Change title format to be more compact
\usepackage{titling}

% Create subtitle command for use in maketitle
\newcommand{\subtitle}[1]{
  \posttitle{
    \begin{center}\large#1\end{center}
    }
}

\setlength{\droptitle}{-2em}
  \title{Summarize Peptide Level Measurements}
  \pretitle{\vspace{\droptitle}\centering\huge}
  \posttitle{\par}
  \author{\href{mailto:WEW@FGCZ.ETHZ.CH}{\nolinkurl{WEW@FGCZ.ETHZ.CH}}}
  \preauthor{\centering\large\emph}
  \postauthor{\par}
  \predate{\centering\large\emph}
  \postdate{\par}
  \date{2018-07-04}


\begin{document}
\maketitle

\subsection{Summarize levels}\label{summarize-levels}

\begin{longtable}[]{@{}lrrrr@{}}
\caption{summary}\tabularnewline
\toprule
NR.Isotope.Label & NR.protein\_Id & NR.peptide\_Id & NR.precursor\_Id &
NR.fragment\_Id\tabularnewline
\midrule
\endfirsthead
\toprule
NR.Isotope.Label & NR.protein\_Id & NR.peptide\_Id & NR.precursor\_Id &
NR.fragment\_Id\tabularnewline
\midrule
\endhead
L/H & 13 & 45 & 45 & 218\tabularnewline
\bottomrule
\end{longtable}

\begin{verbatim}
## [1] 4
\end{verbatim}

\begin{longtable}[]{@{}lrrrrll@{}}
\caption{Protein Summaries}\tabularnewline
\toprule
protein\_Id & peptide\_Id & precursor\_Id & fragment\_Id & nrpeptides &
Precursors & Fragments\tabularnewline
\midrule
\endfirsthead
\toprule
protein\_Id & peptide\_Id & precursor\_Id & fragment\_Id & nrpeptides &
Precursors & Fragments\tabularnewline
\midrule
\endhead
iRT\_Protein & 11 & 11 & 55 & 11 & 1-1 & 5-5\tabularnewline
sp\textbar{}O95477\textbar{}ABCA1\_HUMAN & 3 & 3 & 14 & 3 & 1-1 &
4-5\tabularnewline
sp\textbar{}P02786\textbar{}TFR1\_HUMAN & 3 & 3 & 15 & 3 & 1-1 &
5-5\tabularnewline
sp\textbar{}P04406\textbar{}G3P\_HUMAN & 3 & 3 & 15 & 3 & 1-1 &
5-5\tabularnewline
sp\textbar{}P08195\textbar{}4F2\_HUMAN & 3 & 3 & 15 & 3 & 1-1 &
5-5\tabularnewline
sp\textbar{}P49281\textbar{}NRAM2\_HUMAN & 3 & 3 & 13 & 3 & 1-1 &
4-5\tabularnewline
sp\textbar{}P63104\textbar{}1433Z\_HUMAN & 2 & 2 & 10 & 2 & 1-1 &
5-5\tabularnewline
sp\textbar{}Q01650\textbar{}LAT1\_HUMAN & 3 & 3 & 15 & 3 & 1-1 &
5-5\tabularnewline
sp\textbar{}Q15043\textbar{}S39AE\_HUMAN & 3 & 3 & 15 & 3 & 1-1 &
5-5\tabularnewline
sp\textbar{}Q15365\textbar{}PCBP1\_HUMAN & 3 & 3 & 15 & 3 & 1-1 &
5-5\tabularnewline
sp\textbar{}Q9C0K1\textbar{}S39A8\_HUMAN & 2 & 2 & 10 & 2 & 1-1 &
5-5\tabularnewline
sp\textbar{}Q9NP59\textbar{}S40A1\_HUMAN & 3 & 3 & 11 & 3 & 1-1 &
3-5\tabularnewline
sp\textbar{}Q9UHI5\textbar{}LAT2\_HUMAN & 3 & 3 & 15 & 3 & 1-1 &
5-5\tabularnewline
\bottomrule
\end{longtable}

\begin{figure}
\centering
\includegraphics{HLReport_files/figure-latex/missingFigIntensityHistorgram-1.pdf}
\caption{histogram of mean condition intensities per transition}
\end{figure}

\begin{figure}
\centering
\includegraphics{HLReport_files/figure-latex/missingFigBarplot-1.pdf}
\caption{nr of missing}
\end{figure}

\begin{figure}
\centering
\includegraphics{HLReport_files/figure-latex/missingFigBarplotCumsum-1.pdf}
\caption{cumulative sum of missing}
\end{figure}


\end{document}
